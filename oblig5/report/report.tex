\documentclass[a4paper,11pt]{article}


%\documentclass[journal = ancham]{achemso}
%\setkeys{acs}{useutils = true}
%\usepackage{fullpage}
%\usepackage{natbib}
\pretolerance=2000
\tolerance=6000
\hbadness=6000
%\usepackage[landscape]{geometry}
%\usepackage{pxfonts}
%\usepackage{cmbright}
%\usepackage[varg]{txfonts}
%\usepackage{mathptmx}
%\usepackage{tgtermes}
\usepackage[utf8]{inputenc}
%\usepackage{fouriernc}
%\usepackage[adobe-utopia]{mathdesign}
\usepackage[T1]{fontenc}
%\usepackage[norsk]{babel}
\usepackage{epsfig}
\usepackage{graphicx}
\usepackage{amsmath}
%\usepackage[version=3]{mhchem}
\usepackage{pstricks}
\usepackage[font=small,labelfont=bf,tableposition=below]{caption}
\usepackage{subfig}
%\usepackage{varioref}
\usepackage{hyperref}
\usepackage{listings}
\usepackage{sverb}
%\usepackage{microtype}
%\usepackage{enumerate}
\usepackage{enumitem}
%\usepackage{lineno}
%\usepackage{booktabs}
%\usepackage{changepage}
%\usepackage[flushleft]{threeparttable}
\usepackage{pdfpages}
\usepackage{float}
\usepackage{mathtools}
%\usepackage{etoolbox}
%\usepackage{xstring}

\floatstyle{plaintop}
\restylefloat{table}
%\floatsetup[table]{capposition=top}

\setcounter{secnumdepth}{3}

\newcommand{\tr}{\, \text{tr}\,}
\newcommand{\diff}{\ensuremath{\; \text{d}}}
\newcommand{\sgn}{\ensuremath{\; \text{sgn}}}
\newcommand{\UA}{\ensuremath{_{\uparrow}}}
\newcommand{\RA}{\ensuremath{_{\rightarrow}}}
\newcommand{\QED}{\left\{ \hfill{\textbf{QED}} \right\}}

%\newcommand{\diff}{%
%    \IfEqCase{frac{\diff}{%
%        {\ensuremath{frac{\text{d}} }}%
%        {\ensuremath{\; \text{d}} }% 
%    }[\PackageError{diff}{Problem with diff}{}]%
%}%


\date{\today}
\title{Numerical integration:\\ Two electrons in a harmonic oscillator well\\ \small{Project 5 -- FYS4150}}
\author{Candidate \textbf{105}}


\begin{document}


\onecolumn
\maketitle{}

\begin{abstract}
\end{abstract}

\section{Introduction}

\subsection{Numerical integration}
The expectation value of an observable $\mathbf{O}(x,p)$\footnote{Notation is borrowed from \cite{Griffiths:2005} and \cite{MHJ:2013}.}, where $x$ denotes position and $p$ denotes momentum, can be found as an inner product,
\begin{equation}
    \langle O \rangle   =   \langle \psi | \hat{O} \psi \rangle
    \label{eq:observable}
\end{equation}
where the transition $\mathbf{O} \to \hat{\mathbf{O}}$ denotes that the observable can be found using the corresponding operator on the wave function. Using integral notation, the above expression becomes
\begin{equation}
    \mathbf{O}(p,x)   =   \int\limits_{-\infty}^{+\infty} \diff \mathbf{R}_1 \diff \mathbf{R}_2 \dots \diff \mathbf{R}_N \; \psi^*(\mathbf{R}_1, \mathbf{R}_2, \dots, \mathbf{R}_N ) \; \hat{\mathbf{O}} \; \psi(\mathbf{R}_1, \mathbf{R}_2, \dots, \mathbf{R}_N )
    \label{eq:observable_int1}
\end{equation}
where $\mathbf{R}_i$ represents the coordinates of particle $i$ of the wave function. Above it was assumed that the wave function has been normalised, that is,
\[ \langle \psi | \psi \rangle = \int\limits_{-\infty}^{+\infty}\! \diff \mathbf{R}_1 \diff \mathbf{R}_2 \dots \diff \mathbf{R}_N \; \psi^*(\mathbf{R}_1, \mathbf{R}_2, \dots, \mathbf{R}_N ) \, \psi(\mathbf{R}_1, \mathbf{R}_2, \dots, \mathbf{R}_N = 1 \]
otherwise, the expectation value of the observable $\mathbf{O}$ can be found as
\begin{equation}
    \mathbf{O}(p,x)   =   \frac{\int_{-\infty}^{+\infty} \diff \mathbf{R}_1 \diff \mathbf{R}_2 \dots \diff \mathbf{R}_N \; \psi^*(\mathbf{R}_1, \mathbf{R}_2, \dots, \mathbf{R}_N ) \; \hat{\mathbf{O}} \; \psi(\mathbf{R}_1, \mathbf{R}_2, \dots, \mathbf{R}_N ) }{ \int_{-\infty}^{+\infty} \diff \mathbf{R}_1 \diff \mathbf{R}_2 \dots \diff \mathbf{R}_N \; \psi^*(\mathbf{R}_1, \mathbf{R}_2, \dots, \mathbf{R}_N ) \, \psi(\mathbf{R}_1, \mathbf{R}_2, \dots, \mathbf{R}_N }.
    \label{eq:observable_final}
\end{equation}

In this project, the observable to be found is the energy $E$. From the Schr\"odinger equation it follows that the corresponding operator is the Hamiltonian operator $\hat{\mathbf{H}}$,
\begin{equation}
    \hat{\mathbf{H}} \psi = E \psi
    \label{eq:schrodinger}
\end{equation}
where the Hamiltonian operator is defined as
\begin{equation}
    \hat{\mathbf{H}}    \equiv  -\frac{\hbar^2}{2m} \nabla^2 + V(\mathbf{R}_1, \mathbf{R}_2, \dots, \mathbf{R}_N).
    \label{eq:hamiltonian}
\end{equation}

The standard deviation and variance of the determinate states of an observable $\mathbf{O}$, including the energy, can be found. Here, the corresponding operator of the observable $\mathbf{O}$ is $\hat{\mathbf{O}}$. The variance is
\begin{align*}
    \sigma^2 &= \langle O^2 \rangle - \langle O \rangle^2 \\
    &=  \langle \psi | \hat{O} \hat{O} | \psi \rangle - \langle \psi | \hat{O} \psi \rangle^2 \\
    &=  \langle \psi | \hat{O} q | \psi \rangle - \langle \psi | q | \psi \rangle^2 \\
    &= q \langle \psi | \hat{O} | \psi \rangle - \left( q \langle \psi | \psi \rangle \right)^2 \\
    &= q^2 - \left( q \right)^2 \\
    &= 0
\end{align*}
where it was used that the eigenvalue for the operator $\hat{\mathbf{O}}$ on $\psi$ is $q$, a number which can be moved outside the inner product, and that the wave function is normalised, $\langle \psi | \psi \rangle = 1$.

\subsection{Quantum mechanical cases}
The quantum mechanical cases of this project will be two electrons interacting in a harmonic oscillator well, with wave function
\begin{equation}
    \psi_{T1}(\mathbf{r}_1, \mathbf{r}_2) = \exp \left[ -\alpha^2 \left( r_1^2 + r_2^2  \right)/2 \right]
    \label{eq:wavefuncT1}
\end{equation}
and potential
\begin{equation}
    V(\mathbf{r}_1, \mathbf{r}_2) = \frac{1}{2} \left( r_1^2 + r_2^2 \right) + \frac{1}{|\mathbf{r}_1 - \mathbf{r}_2 | }
    \label{eq:potential}
\end{equation}
where 
\[ \mathbf{r}_i \equiv x_i \mathbf{e}_x + y_i \mathbf{e}_y + z_i \mathbf{e}_z. \]

Another case which will be studied is an approach to the helium atom where the wave function will resemble 
\begin{equation}
    \psi_{T2}(\mathbf{r}_1, \mathbf{r}_2, \mathbf{r}_{12}) = \exp \left[ -\alpha^2 \left( r_1^2 - r_2^2 \right)/2 \right] \exp \left[ \frac{r_{12}}{2(1 + \beta r_{12})} \right]
    \label{eq:wavefuncT2}
\end{equation}
where $\mathbf{r}_{12} \equiv \mathbf{r}_1 - \mathbf{r}_2$, and the potential is the same as in eq.~(\ref{eq:potential}).

This project will use different methods to determine observables defined in eq.~(\ref{eq:observable_final}). Gaussian quadrature, with orthogonal Legendre and Hermite polynomials are used as well as brute force Monte Carlo sampling and importance sampling from a normal distribution. 

This project will also apply variational Monte Carlo techniques for both quantum mechanical cases, where the factors $\alpha$ and $\beta$ are varied and the Metropolis algorithm is applied to create the distribution. 

\section{Methods}
Gaussian quadrature, brute force Monte Carlo and importance sampling Monte Carlo are applied to find the expectation value of the correlation energy between two electrons, given as
\begin{align}
    \langle \frac{1}{| \mathbf{r}_1 - \mathbf{r}_2 |} \rangle &= \int\limits_{-\infty}^{+\infty} \! \diff \mathbf{r}_1 \diff \mathbf{r}_2 \; \psi_{T1}^* \; \frac{1}{|\mathbf{r}_1 - \mathbf{r}_2 |} \; \psi_{T1} \notag \\
    &= \int\limits_{-\infty}^{+\infty} \! \diff \mathbf{r}_1 \diff \mathbf{r}_2 \; \exp\left[ -\alpha \left( r_1^2 + r_2^2 \right) \right] \frac{1}{|\mathbf{r}_1 - \mathbf{r}_2 |}.
    \label{eq:exp_corr_energy}
\end{align}

Variational Monte Carlo is applied to find the expectation value for the Hamiltonian operator for the wave functions $\psi_{T1}$ and $\psi_{T2}$. 

\subsection{Gaussian quadrature}
Gaussian quadrature differs from other quadrature methods, or methods of numerical integration, in the way the way the weights $w_i$ and integration points $x_i$ are chosen in the approximation
\begin{equation}
    \int_{a}^{b} f(x) \diff x \approx \sum_i f(x_i) w_i
    \label{eq:numint}
\end{equation}
where choices of $x_i$ and $w_i$ for two other quadrature methods are:
\begin{description}
    \item[Midpoint method] The integral is approximated through a subdivision into $N$ intervals of equal length $h = (b-a)/N$ which work as the weights $w_i = h$, whereas the integration points are the midpoints of each interval, $x_i = a + (i/2) h \equiv x_{i-1/2}$ for $i=1,2,\dots,N$. The local error goes as $O(h^3)$ whereas the global error goes as $O(h^2)$.
    \item[Trapezoidal method] The function to be integrated is evaluated at $N$ points $x_i = a + ih$ with weights $w_i = h$ for $i=2,3,\dots,N-1$ whereas for $i=1,N$, $w_i = h/2$. The local error goes as $O(h^3)$ and the global error goes as $O(h^2)$. 
\end{description} 

In the case of Gaussian quadrature, the integration points $x_i$ are chosen as the locations of the zeros of an orthogonal polynomial of order $N$ which is orthogonal for a certain range $[p_i, p_f]$.  

The choice of orthogonal polynomial reflects both the integration range, and the form of the integrand -- it is useful to consider a rewrite where the integrand $f(x) \to W(x) g(x)$ with $W(x)$ the \textit{weight function} that project the function $g(x)$ onto the chosen orthogonal basis of polynomials. 

In the case where $W(x)$ can be singled out, the Gaussian quadrature consists of evaluating $g(x_i)$ at the zeros of the chosen orthogonal polynomial and multiplying with the weights $w_i$.

To determine the weights $w_i$, a closer look on the polynomial representation of the initial integrand $f(x)$ is required. In Gaussian quadrature, a function of order $N$ can be represented by a polynomial $P_{2N-1}(x)$ of order $2N-1$. This can, in turn, be represented by the orthogonal polynomials $L_N(x)$, $P_{N-1}(x)$ and $Q_{N-1}(x)$,
\begin{equation}
    \int f(x) \diff x \approx \int P_{2N-1}(x) \diff x = \int L_N(x) P_{N-1}(x) + Q_{N-1}(x) \diff x
    \label{eq:quadrature}
\end{equation}
where $L_N(x)$ is the chosen orthogonal polynomial giving the basis for the function space (for a given range of $x$). As $P_{N-1}(x)$ is orthogonal to $L_N(x)$ (the differences in order $N-1$ and $N$ lead to the product being zero, from eqs.~(5.10, 5.14) in \cite{MHJ:2013}), the integral can be written
\[ \int f(x) \diff x \approx \int Q_{N-1}(x) \diff x = 2 \sum_{i=0}^{N-1} \left( L^{-1} \right)_{0i} P_{2N-1}(x_i) \]
(from eqs.~(5.15, 5.16, 5,17) in \cite{MHJ:2013}) where the weigths are determined as $w_i = 2 \left( L^{-1} \right)_{0i}$.

Table (\ref{tab:polys}) summarises possible weight functions, orthogonality ranges and orthogonal polynomials (from \cite{MHJ:2013}).

\begin{table}
    \centering
    \caption{Relevant weight functions that project a given function $g(x)$ onto a basis of orthogonal polynomials that are orthogonal for the specified interval. From \cite{MHJ:2013}.}
    \begin{tabular}{l l l}
        \hline 
        Weight function & Interval, $x\in[a,b]$ & Orthogonal polynomial \\
        \hline
        $W(x) = 1$          & $[-1,1]$              & Legendre \\
        $W(x) = e^{-x^2}$   & $(-\infty, +\infty)$  & Hermite \\
        $W(x) = x^\alpha e^{-x}$ & $[0, +\infty)$   & Laguerre \\
        $W(x) = 1/\sqrt{1 - x^2}$& $[-1, 1]$        & Chebyshev \\
        \hline
    \end{tabular}
    \label{tab:polys}
\end{table}

\subsection{Monte Carlo methods}
A 

\section{Results}
\subsection{Gauss-Legendre quadrature}

\begin{table}
    \centering
    \caption{Gauss-Legendre quadrature results for the correlation energy between two interacting electrons in a harmonic oscillator potential well. Note that $N=128$ required $128^6 \approx 4400$ billion integration points and required more than three hours of computations on the 64-core computing node \texttt{nekkar.uio.no}, fully parallelised using OpenMP.}
    \begin{tabular}{l c}
        \hline
        $N$ & $I = \langle 1/|\mathbf{r}_1 - \mathbf{r}_2| \rangle$ \\
        \hline
        8   &   5.74062 \\
        16  &   19.9532 \\
        32  &   23.4625 \\
        64  &   24.4124 \\
        128 &   24.6569 \\
        \hline
    \end{tabular}
    \label{tab:<+label+>}
\end{table}<++>

\begin{table}
    \centering
    \caption{Gauss-Hermite quadrature results for the correlation energy between two interacting electrons in a harmonic oscillator potential well.}
    \begin{tabular}{l c}
        \hline
        $N$ &   $I = \langle 1/|\mathbf{r}_1 - \mathbf{r}_2| \rangle$ \\
        \hline
        8   &   21.5379 \\
        16  &   23.0806 \\
        32  &   23.8942 \\
        64  &   24.3128 \\
        128 &   24.5251 \\
        \hline
    \end{tabular}
    \label{tab:<+label+>}
\end{table}<++>

\subsection{Monte Carlo}

\subsubsection{Brute force}

\begin{table}
    \centering
    \caption{Brute force Monte Carlo integration results.}
    \begin{tabular}{l l l l}
        \hline
        $N$         &   Results     & Std.dev $\sigma$  & Wall time elapsed  \\
        \hline
        $2^{22}$    & 24.3568145555 & 1.1251229631   & 1.603s \\
        $2^{23}$    & 24.459612255  & 0.948967998187 & 3.14832s \\
        $2^{28}$    & 24.823832143  & 0.172828762223 & 92.1901s \\
        \hline
    \end{tabular}
    \label{tab:res_bruteforceMC}
\end{table}
\begin{table}
    \centering
    \caption{Importance sampling Monte Carlo integration results.}
    \begin{tabular}{l l l l}
        \hline
        $N$         &   Results     & Std.dev $\sigma$ & Wall time elapsed  \\
        \hline
        $2^{22}$    & 24.7308400577 & 0.00903293421986 & 14.8966s \\
        $2^{23}$    & 24.7369331549 & 0.00644995146458 & 30.1007s \\
        $2^{28}$    & 24.734050648  & 0.00113551436761 & 1033.96s \\
        \hline
    \end{tabular}
    \label{tab:res_importanceMC}
\end{table}

\subsection{Variational Monte Carlo}
\begin{figure}[htb]
    \centering
    \includegraphics[width=0.8\columnwidth]{fig/fig1_T1_E.pdf}
    \caption{The expectation value for the energy, $\langle E \rangle$ plotted against the variational parameter $\alpha$ in the wave function $\psi_{T1}$. The solid blue line represents $N=2^{17}$ Monte Carlo cycles, and the dashed red line represents $N=2^{24}$ Monte Carlo cycles. See fig.~(\ref{fig:2_t1_17_sigma}) for a plot of the corresponding standard deviations. The minimum for the expectation value is $\langle E \rangle = 2.2307$ at $\alpha = -0.96$ for $N=2^{24}$.}
    \label{fig:2_t1_17_E}
\end{figure}

\begin{figure}[htpb]
    \centering
    \includegraphics[width=0.8\columnwidth]{fig/fig2_T1_sigma.pdf}
    \caption{Standard deviation for the local energy plotted against $\alpha$, for the case where the trial wave function is $\psi_{T1}$ with $N=2^{17}, 2^{24}$ Monte Carlo cycles. }
    \label{fig:2_t1_17_sigma}
\end{figure}<++>

\section{Discussion and conclusion}



\bibliography{referanser}
\bibliographystyle{plain}


\end{document}

