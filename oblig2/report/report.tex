\documentclass[a4paper,11pt]{article}


%\documentclass[journal = ancham]{achemso}
%\setkeys{acs}{useutils = true}
%\usepackage{fullpage}
%\usepackage{natbib}
\pretolerance=2000
\tolerance=6000
\hbadness=6000
%\usepackage[landscape]{geometry}
%\usepackage{pxfonts}
%\usepackage{cmbright}
%\usepackage[varg]{txfonts}
%\usepackage{mathptmx}
%\usepackage{tgtermes}
\usepackage[utf8]{inputenc}
%\usepackage{fouriernc}
%\usepackage[adobe-utopia]{mathdesign}
\usepackage[T1]{fontenc}
%\usepackage[norsk]{babel}
\usepackage{epsfig}
\usepackage{graphicx}
\usepackage{amsmath}
%\usepackage[version=3]{mhchem}
\usepackage{pstricks}
\usepackage[font=small,labelfont=bf,tableposition=below]{caption}
\usepackage{subfig}
%\usepackage{varioref}
\usepackage{hyperref}
\usepackage{listings}
\usepackage{sverb}
%\usepackage{microtype}
%\usepackage{enumerate}
\usepackage{enumitem}
%\usepackage{lineno}
%\usepackage{booktabs}
%\usepackage{changepage}
%\usepackage[flushleft]{threeparttable}
\usepackage{pdfpages}
\usepackage{float}
\usepackage{mathtools}

\floatstyle{plaintop}
\restylefloat{table}
%\floatsetup[table]{capposition=top}

\setcounter{secnumdepth}{0}

\newcommand{\tr}{\, \text{tr}\,}
\newcommand{\diff}{\ensuremath{\; \text{d}}}
\newcommand{\sgn}{\ensuremath{\; \text{sgn}}}
\newcommand{\UA}{\ensuremath{_{\uparrow}}}
\newcommand{\RA}{\ensuremath{_{\rightarrow}}}
\newcommand{\QED}{\left\{ \hfill{\textbf{QED}} \right\}}


\date{\today}
\title{Schr\"{o}dinger equation for two electrons\\ \small{Project 2 -- FYS4150}}
\author{Marius Berge Eide \\
\texttt{mariusei@astro.uio.no}}


\begin{document}


\onecolumn
\maketitle{}

\begin{abstract}
    
\end{abstract}<++>

\section{Introduction}
The time independent Schr\"{o}dinger equation can give the different energy levels in a spherical symmetric case with a given wave function $R(r)$ in a potential $V(r)$, where $r$ is the radial coordinate,

\begin{equation}
    -\frac{\hbar^2}{2m} \left( \frac{1}{r^2} \frac{\diff}{\diff r} r^2 \frac{\diff}{\diff r} - \frac{l\left( l+1 \right)}{r^2}  \right) R(r) + V(r) R(r) = ER(r),
    \label{eq:schrodinger}
\end{equation}

with $l=0,1,2,\dots$ the azimuthal quantum numbers or \textit{orbital angular momentum quantum numbers}.

In the harmonic oscillator potential $V(r) = (1/2)kr^2$ with $k = m \omega^2$ where $m$ is the mass of the particle and $\omega$ the oscillator frequency, the energies are given as
\begin{equation}
    E_{nl} = \hbar \omega \left( 2n + l + \frac{3}{2} \right)
    \label{eq:energies}
\end{equation}
where $n=0,1,2,\dots$ are the levels of excitation.

Rewriting the wave equation $R(r) = (1/r) u(r)$, the second derivative can be written 

\begin{align*}
    \frac{\diff}{\diff r} R(r) &= \frac{\diff}{\diff r} \left[ \frac{1}{r} u(r) \right] \\
    &= -\frac{1}{r^2}u(r) + \frac{1}{r} \frac{\diff u(r)}{\diff r} \\
    \frac{\diff}{\diff r} \left[ r^2 \frac{\diff R(r)}{\diff r} \right] 
    &= \frac{\diff}{\diff r} \left[ -u(r) + r \frac{\diff u(r)}{\diff r} \right] \\
    &= -\frac{\diff u(r)}{\diff r} + \left[ \frac{\diff u(r)}{\diff r} \frac{\diff r}{\diff r} + r \frac{\diff^2 u(r)}{\diff r^2} \right] \\
    &= +r \frac{\diff^2 u(r)}{\diff r^2},
\end{align*}
and substituting $r \to \alpha \rho$ where $\rho$ is a dimensionless variable, the operator becomes

\[ \frac{\diff}{\diff r} = \frac{\diff \rho}{\diff r} \frac{\diff}{\diff \rho} = \frac{1}{\alpha} \frac{\diff}{\diff \rho}, \]
and assuming $l=0$ for this project, the time-independent Schr\"{o}dinger equation becomes

\begin{align}
    -\frac{\hbar^2}{2m\alpha^2} \frac{\diff^2 u(\rho)}{\diff \rho^2} + \frac{1}{2} k \alpha^2 \rho^2 u(\rho) &= E u(\rho) \label{eq:se0} \\
    -\frac{\diff^2 u(\rho)}{\diff \rho^2} + \frac{2m \alpha^2}{\hbar^2}\frac{1}{2} k \alpha^2 \rho^2 u(\rho) &= \frac{2m\alpha}{\hbar^2} E u(\rho) \label{eq:se1} \\
    -\frac{\diff^2 u}{\diff \rho^2} + \frac{mk}{\hbar^2}\alpha^4 \rho^2 u(\rho) &= \lambda u(\rho) \label{eq:se2} \\
    -\frac{\diff^2 u}{\diff \rho^2} + \rho^2 u(\rho) &= \lambda u(\rho) \label{eq:se3}
\end{align}
where, in step (\ref{eq:se1}) $\to$ (\ref{eq:se2}), the constants were set to one,
\[ \frac{mk}{\hbar^2}\alpha^4 = 1 \quad \Rightarrow \quad \alpha = \left( \frac{\hbar^2}{mk}  \right)^{^1\! / _4}, \]
and the eigenvalues were defined as
\begin{equation}
    \lambda \equiv \frac{2m\alpha^2}{\hbar^2} E. 
    \label{eq:eigvals}
\end{equation}

The given rewrite of the time independent Schr{o}dinger equation lays the foundation for using numerical methods to find the solutions as eigenvalue-problems.

\section{Methods}
\subsection{Tridiagonalisation}
The solutions of the time-independent Schr\"{o}dinger equation can be found using concepts from project~1.

The right hand side of eq.~(\ref{eq:se3}) has the second order differential operator which can be approximated as
\[ -\frac{u\left( \rho_i + h \right) - 2u(\rho_i) + u\left( \rho_i - h \right)}{h} \equiv -\frac{u_{i+1} - 2u_i + u_{i-1}}{h^2}, \]
where $h = (\rho_{\rm max} - \rho_{\rm min})/(N-1)$ is the step length with $N$ steps. 

The potential term can be rewritten,
\[ \rho_i^2 u_i \equiv V_i u_i, \]
and the right hand side terms for appropriate steps $\rho_i$ can be collected,
\begin{align*}
    \frac{2}{h^2} u_i + \rho_i^2 u_i = \left( \frac{2}{h^2} + V_i \right) u_i &= d_i u_i \\
    -\frac{u_{i+1}}{h^2} &= e_i u_{i+1} \\
    -\frac{u_{i-1}}{h^2} &= e_i u_{i-1},
\end{align*}
making it possible to rewrite eq.~(\ref{eq:se3}) as
\[ d_i u_i + e_{i-1} u_{i-1} + e_{i+1} u_{i+1} = \lambda u_i, \]
where it should be noted that $e_i$ remains a constant, meaning those terms give rise to the symmetry of the matrix equivalent of the problem,
\begin{equation}
    \begin{bmatrix}
        d_1     & e_1   & 0     &       &       &  \\
        e_1     & d_2   & e_2   & 0     &       &  \\
        0       & e_2   & d_3   & e_3   & 0     &  \\
                & 0     &\ddots & \ddots& \ddots& 0\\
                &       & 0     &e_{N-3}&d_{N-2}&e_{N-2} \\
                &       &       & 0     &e_{N-2}&d_{N-1}
    \end{bmatrix}
    \begin{bmatrix}
        u_1 \\ u_2 \\ u_3 \\ \vdots \\ u_{N-2} \\ u_{N-1}
    \end{bmatrix}
     =
    \lambda
    \begin{bmatrix}
        u_1 \\ u_2 \\ u_3 \\ \vdots \\ u_{N-2} \\ u_{N-1}
    \end{bmatrix}
    \label{eq:se_matrix}
\end{equation}
or, simply
\begin{equation}
    A \mathbf{u} = \lambda \mathbf{u}.
    \label{eq:se_vector}
\end{equation}


\subsection{Applying the Jacobi eigenvalue algorithm}
The Jacobi eigenvalue algorithm can be applied on the symmetric matrix of eq.~(\ref{eq:se_matrix}) in order to find the eigenvalues $\lambda_n$ for the various energy levels $n$, which are related to the observable energy $E$ through eq.~(\ref{eq:energies}). 

The central feature of the Jacobi eigenvalue algorithm is to zero out the greatest value off the diagonal, \texttt{max($|A_{k,l,k\neq l}|$)} using Givens rotation.

Givens rotation applies to the plane spanned by the coordinate axes $k,l$ in $N$-dimensional space, and can be expressed using the matrix $P_{k,l}$ where the subscript does not denote indices, but which rows and columns it will affect:
\begin{equation}
    P_{k,l} =
    \begin{bmatrix}
        1   & 0         &\dots  &           & 0 \\
        0   & \cos\theta&       &+\sin\theta&   \\
    \vdots  &           &1      &           & \vdots  \\
            &-\sin\theta&       & \cos\theta& 0 \\
        0   &           &\dots  & 0         & 1
    \end{bmatrix}.
    \label{eq:givens}
\end{equation}

The Givens matrix is orthogonal, $P_{k,l}^T P_{k,l} = I$, and a series of similarity transformations $P_{k,l}^T P_{k,l}^T \cdots P_{k,l}^T A P_{k,l} \cdots P_{k,l}  P_{k,l}$ will cancel out all off-diagonal elements, with $P_{k,l} A$ only affecting the columns, and $A P_{k,l}$ only affecting the rows \cite{numrecipes}. For each transformation $A' = P_{k,l}^T A P_{k,l} $, the new elements in $A'$ are
\begin{align*}
    A_{jj}' &= A_{jj}   \qquad j\neq k, j \neq l \\
    A_{jk}' &= A_{jk} \cos \theta - A_{jl} \sin\theta   \qquad j\neq k, j \neq l \\
    A_{jl}' &= A_{jk} \cos \theta + A_{jl} \sin\theta   \qquad j\neq k, j \neq l \\
    A_{kk}' &= A_{kk} \cos^2\theta - 2 A_{kl} \cos\theta \sin\theta 
                    ~+ A_{ll} \sin^2\theta \\
    A_{ll}' &= A_{ll} \cos^2\theta + 2 A_{kl} \cos\theta \sin\theta 
                    ~+ A_{ll} \sin^2\theta \\
    A_{kl}' &= \left( A_{kk} - A_{ll} \right) \cos\theta \sin\theta 
                + A_{kl} \left( \cos^2\theta -\sin^2\theta \right)
\end{align*}
where it is required that element $A_{kl}'$ becomes zero, giving
\begin{align*}
    0   &= \left( A_{kk} - A_{ll} \right) \cos\theta \sin\theta 
                + A_{kl} \left( \cos^2\theta -\sin^2\theta \right) \\
    \frac{A_{ll} - A_{kk}}{A_{kl}} &= \frac{\cos^2\theta - \sin^2\theta}{\cos\theta \sin\theta}, \\
\end{align*}
using two geometric identities,
\begin{align*}
    \sin 2\theta &= 2\sin\theta \cos\theta \\
    \cos^2\theta - \sin^2 \theta &= \cos 2\theta 
\end{align*}
the null-expression for $A_{kl}'$ can be rewritten, now denoting $\cos\theta = c$, $\sin\theta  = s$, $\tan \theta \equiv \sin\theta/\cos\theta = t$, and also $\sin 2\theta = s(2\theta)$ and $\sin 2\theta = s\left( 2\theta \right)$, giving
\begin{align*}
    \frac{A_{kk} - A_{ll}}{A_{kl}} &= - \frac{c^2 - s^2}{cs} \\
    &= -2 \frac{c\left( 2\theta \right)}{s\left( 2\theta \right)} \\
    \frac{A_{ll} - A_{kk}}{2A_{kl}} &= \cot 2\theta \equiv \tau.
\end{align*}

Furthermore, to find the tangent, the identity
\[ \tan 2\theta = \frac{2 t}{1 - t^2} = \frac{1}{\tau} \]
can be solved for $t$, from the quadratic function,
\begin{align*}
    \frac{2t}{1 - t^2} &= \frac{1}{\tau}  \\
    2t\tau &= 1 - t^2 \\
    t^2 + 2t \tau - 1 &= 0
\end{align*}
with solutions $t = -\tau \pm \sqrt{\tau^2 + 1}$. This latter equation should be rewritten for numerically stability,
\begin{align*}
    t &= \frac{\left( -\tau \pm \sqrt{\tau^2 + 1}\right) \left( -\tau \mp \sqrt{\tau^2 + 1} \right)}{-\tau \mp \sqrt{\tau^2 + 1} } \\
    &= \frac{-1}{-\tau \mp \sqrt{\tau^2 + 1}}
\end{align*}
where the smaller value should be chosen, as to minimise the difference
\[ ||A' - A||_F^2 = 4\left( 1-c \right) \sum\limits_{\substack{j=1\\j\neq k,l}}^N \left( A_{ik}^2 + A_{il}^2 \right) + \frac{2A^2_{kl}}{c^2}. \]

The smaller $|\tan\theta|$ ensures a larger $\cos\theta$ which would minimise the difference given above, and according to \cite{numrecipes}, the expression for the tangent can be written
\begin{equation}
    t = \frac{\sgn\left( \tau \right)}{|\tau| + \sqrt{\tau^2 + 1} },
    \label{eq:t}
\end{equation}
with $\sgn()$ the signum function, returning the sign of the variable.

\subsection{Two electrons in a potential well}
With two electrons in a harmonic oscillator potential well that also interact via repulsive Coulomb interaction, the time independent Schr\"{o}dinger equation becomes
\begin{equation}
    \left( -\frac{\hbar^2}{2m} \frac{\diff}{\diff r_1^2} - \frac{\hbar^2}{2m} \frac{\diff^2}{\diff r_2^2} + \frac{1}{2} kr_1^2 + \frac{1}{2} kr_2^2 \right) u\left( r_1, r_2 \right) = E^{(2)} u\left( r_1, r_2 \right)
    \label{eq:se2el}
\end{equation}
where the superscript $(2)$ and coordinates $(r_1, r_2)$ denotes that there is a two-electron wave function. With the relative coordinate $r = r_1 - r_2$, and the coordinate relative to the center of mass,
\[ R = \frac{1}{2} \left( r_1 + r_2 \right) \]
the time independent Schr\"{o}dinger equation becomes
\[ \left( -\frac{\hbar^2}{m}\frac{\diff^2}{\diff r^2} - \frac{\hbar^2}{4m}\frac{\diff^2}{\diff R^2} + \frac{1}{4}kr^2 + kR^2 \right) u(r,R) = E^{(2)} u(r,R). \]

Separating the wave-function $u(r,R) = \psi(r) \phi(R)$ and considering only the first part, $\psi(r)$, in relative coordinates, 
\[ \left( -\frac{\hbar^2}{m} \frac{\diff^2}{\diff r^2} + \frac{1}{4} kr^2 \right) \psi(r) = E_r \psi(r) \]
and modifying the potential by adding repulsive Coulomb interaction between the two electrons,
\[ V(r_1, r_2) = \frac{\beta e^2}{|r_1 - r_2 |} = \frac{\beta e^2}{r} \]
where $\beta e^2 = 1.44$ eV nm, making the Schr\"{o}dinger equation become
\[ \left( -\frac{\hbar^2}{m} \frac{\diff^2}{\diff r^2} + \frac{1}{4} kr^2 + \frac{\beta e^2}{r} \right) \psi(r) = E_r \psi(r). \]

A change of variable $r =\rho\alpha$ makes the equation become dimensionless, by requiring a fix on the constants,
\[ \frac{m\alpha \beta e^2}{\hbar^2} = 1 \quad \Rightarrow \quad \alpha = \frac{\hbar^2}{m \beta e^2} \]
and introducing a '\textit{frequency}' that describes the strength of the potential well,
\[ \omega_r^2 = \frac{1}{4} \frac{mk}{\hbar^2} \alpha^4, \]
the dimensionless Schr\"{o}dinger equation for relative coordinates $r$ for two interacting electrons becomes
\begin{equation}
    -\frac{\diff^2}{\diff \rho^2} \psi (\rho) + \omega_r^2 \rho^2 \psi(\rho) = \lambda \psi(\rho)
    \label{eq:se_2electrons}
\end{equation}
which can be solved numerically by replacing the potential for the single electron with $V_i = \rho^2 \to V_i' = \omega_r^2 \rho^2 + 1/\rho$. 

\section{Results}

\section{Discussion and conclusion}

\bibliography{referanser}
\bibliographystyle{plain}


\end{document}

