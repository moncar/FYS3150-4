\documentclass[a4paper,11pt]{article}


%\documentclass[journal = ancham]{achemso}
%\setkeys{acs}{useutils = true}
%\usepackage{fullpage}
%\usepackage{natbib}
\pretolerance=2000
\tolerance=6000
\hbadness=6000
%\usepackage[landscape]{geometry}
%\usepackage{pxfonts}
%\usepackage{cmbright}
%\usepackage[varg]{txfonts}
%\usepackage{mathptmx}
%\usepackage{tgtermes}
\usepackage[utf8]{inputenc}
%\usepackage{fouriernc}
%\usepackage[adobe-utopia]{mathdesign}
\usepackage[T1]{fontenc}
%\usepackage[norsk]{babel}
\usepackage{epsfig}
\usepackage{graphicx}
\usepackage{amsmath}
%\usepackage[version=3]{mhchem}
\usepackage{pstricks}
\usepackage[font=small,labelfont=bf,tableposition=below]{caption}
\usepackage{subfig}
%\usepackage{varioref}
\usepackage{hyperref}
\usepackage{listings}
\usepackage{sverb}
%\usepackage{microtype}
%\usepackage{enumerate}
\usepackage{enumitem}
%\usepackage{lineno}
%\usepackage{booktabs}
%\usepackage{changepage}
%\usepackage[flushleft]{threeparttable}
\usepackage{pdfpages}
\usepackage{float}
\usepackage{mathtools}
%\usepackage{etoolbox}
%\usepackage{xstring}

\floatstyle{plaintop}
\restylefloat{table}
%\floatsetup[table]{capposition=top}

\setcounter{secnumdepth}{3}

\newcommand{\tr}{\, \text{tr}\,}
\newcommand{\diff}{\ensuremath{\; \text{d}}}
\newcommand{\sgn}{\ensuremath{\; \text{sgn}}}
\newcommand{\UA}{\ensuremath{_{\uparrow}}}
\newcommand{\RA}{\ensuremath{_{\rightarrow}}}
\newcommand{\QED}{\left\{ \hfill{\textbf{QED}} \right\}}

%\newcommand{\diff}{%
%    \IfEqCase{frac{\diff}{%
%        {\ensuremath{frac{\text{d}} }}%
%        {\ensuremath{\; \text{d}} }% 
%    }[\PackageError{diff}{Problem with diff}{}]%
%}%


\date{\today}
\title{Numerical integration:\\ Two electrons in a harmonic oscillator well\\ \small{Project 5 -- FYS4150}}
\author{Candidate \textbf{105}}


\begin{document}


\onecolumn
\maketitle{}

\begin{abstract}
\end{abstract}

\section{Introduction}

\subsection{Numerical integration}
To calculate the expectation value of an observable $\mathbf{O}(x,p)$, where $x$ denotes position and $p$ denotes momentum,
\begin{equation}
    \langle O \rangle   =   \langle \psi | \hat{O} \psi \rangle
    \label{eq:observable}
\end{equation}
where the transition $\mathbf{O} \to \hat{\mathbf{O}}$ denotes that the observable can be found using the corresponding operator on the wave function. Using integral notation, the above expression becomes
\begin{equation}
    \mathbf{O}(p,x)   =   \int\limits_{-\infty}^{+\infty} \diff \mathbf{R}_1 \diff \mathbf{R}_2 \dots \diff \mathbf{R}_N \; \psi^*(\mathbf{R}_1, \mathbf{R}_2, \dots, \mathbf{R}_N ) \; \hat{\mathbf{O}} \; \psi(\mathbf{R}_1, \mathbf{R}_2, \dots, \mathbf{R}_N )
    \label{eq:observable_int1}
\end{equation}
where $\mathbf{R}_i$ represents the coordinates of particle $i$ of the wave function. Above it was assumed that the wave function has been normalised, that is,
\[ \langle \psi | \psi \rangle = \int\limits_{-\infty}^{+\infty}\! \diff \mathbf{R}_1 \diff \mathbf{R}_2 \dots \diff \mathbf{R}_N \; \psi^*(\mathbf{R}_1, \mathbf{R}_2, \dots, \mathbf{R}_N ) \, \psi(\mathbf{R}_1, \mathbf{R}_2, \dots, \mathbf{R}_N = 1 \]
otherwise, the expectation value of the observable $\mathbf{O}$ can be found as
\begin{equation}
    \mathbf{O}(p,x)   =   \frac{\int_{-\infty}^{+\infty} \diff \mathbf{R}_1 \diff \mathbf{R}_2 \dots \diff \mathbf{R}_N \; \psi^*(\mathbf{R}_1, \mathbf{R}_2, \dots, \mathbf{R}_N ) \; \hat{\mathbf{O}} \; \psi(\mathbf{R}_1, \mathbf{R}_2, \dots, \mathbf{R}_N ) }{ \int_{-\infty}^{+\infty} \diff \mathbf{R}_1 \diff \mathbf{R}_2 \dots \diff \mathbf{R}_N \; \psi^*(\mathbf{R}_1, \mathbf{R}_2, \dots, \mathbf{R}_N ) \, \psi(\mathbf{R}_1, \mathbf{R}_2, \dots, \mathbf{R}_N }.
    \label{eq:observable_final}
\end{equation}

In this project, the observable to be found is the energy $E$. From the Schr\"odinger equation it follows that the corresponding operator is the Hamiltonian operator $\hat{\mathbf{H}}$,
\begin{equation}
    \hat{\mathbf{H}} \psi = E \psi
    \label{eq:schrodinger}
\end{equation}
where the Hamiltonian operator is defined as
\begin{equation}
    \hat{\mathbf{H}}    \equiv  -\frac{\hbar^2}{2m} \nabla^2 + V(\mathbf{R}_1, \mathbf{R}_2, \dots, \mathbf{R}_N).
    \label{eq:hamiltonian}
\end{equation}

The standard deviation and variance of the determinate states of an observable $Q$, including the energy, can be found. Here, the corresponding operator of the observable $Q$ is $\hat{Q}$.
\begin{align*}
    \sigma^2 &= \langle Q^2 \rangle - \langle Q \rangle^2 \\
            &=  \langle \psi | \hat{Q}^2 | \psi \rangle - 
\end{align*}<++>

\section{Methods}
\subsection{Gauss-Legendre}
The Gauss-Legendre quadrature, o


\section{Results}
\subsection{Gauss-Legendre quadrature}

\begin{table}
    \centering
    \caption{Gauss-Legendre quadrature integration results. Note that $N=128$ required $128^6 \approx 4400$ billion integration points and required more than three hours of computations on the 64-core computing node \texttt{nekkar.uio.no}, fully parallelised using OpenMP.}
    \begin{tabular}{l l}
        \hline
        $N$ &   Results \\
        \hline
        8   &   5.74062 \\
        16  &   19.9532 \\
        32  &   23.4625 \\
        64  &   24.4124 \\
        128 &   24.6569 \\
        \hline
    \end{tabular}
    \label{tab:<+label+>}
\end{table}<++>

\begin{table}
    \centering
    \caption{Gauss-Hermite quadrature integration results.}
    \begin{tabular}{l l}
        \hline
        $N$ &   Results \\
        \hline
        8   &   21.5379 \\
        16  &   23.0806 \\
        32  &   23.8942 \\
        64  &   24.3128 \\
        128 &   24.5251 \\
        \hline
    \end{tabular}
    \label{tab:<+label+>}
\end{table}<++>

\subsection{Monte Carlo}

\subsubsection{Brute force}

\begin{table}
    \centering
    \caption{Brute force Monte Carlo integration results.}
    \begin{tabular}{l l}
        \hline
        $N$ &   Results \\
        \hline
        $2^{22}$  & 25.8037959241 \\
        $2^{24}$  & 24.7029013387 \\
        $2^{29}$  & 24.7242439051 \\
        $2^{30}$  & 24.85477499   \\ 
        \hline
    \end{tabular}
    \label{tab:<+label+>}
\end{table}<++>

\section{Discussion and conclusion}

\bibliography{referanser}
\bibliographystyle{plain}


\end{document}

