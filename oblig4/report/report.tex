\documentclass[a4paper,11pt]{article}


%\documentclass[journal = ancham]{achemso}
%\setkeys{acs}{useutils = true}
%\usepackage{fullpage}
%\usepackage{natbib}
\pretolerance=2000
\tolerance=6000
\hbadness=6000
%\usepackage[landscape]{geometry}
%\usepackage{pxfonts}
%\usepackage{cmbright}
%\usepackage[varg]{txfonts}
%\usepackage{mathptmx}
%\usepackage{tgtermes}
\usepackage[utf8]{inputenc}
%\usepackage{fouriernc}
%\usepackage[adobe-utopia]{mathdesign}
\usepackage[T1]{fontenc}
%\usepackage[norsk]{babel}
\usepackage{epsfig}
\usepackage{graphicx}
\usepackage{amsmath}
%\usepackage[version=3]{mhchem}
\usepackage{pstricks}
\usepackage[font=small,labelfont=bf,tableposition=below]{caption}
\usepackage{subfig}
%\usepackage{varioref}
\usepackage{hyperref}
\usepackage{listings}
\usepackage{sverb}
%\usepackage{microtype}
%\usepackage{enumerate}
\usepackage{enumitem}
%\usepackage{lineno}
%\usepackage{booktabs}
%\usepackage{changepage}
%\usepackage[flushleft]{threeparttable}
\usepackage{pdfpages}
\usepackage{float}
\usepackage{mathtools}
%\usepackage{etoolbox}
%\usepackage{xstring}

\floatstyle{plaintop}
\restylefloat{table}
%\floatsetup[table]{capposition=top}

\setcounter{secnumdepth}{3}

\newcommand{\tr}{\, \text{tr}\,}
\newcommand{\diff}{\ensuremath{\; \text{d}}}
\newcommand{\sgn}{\ensuremath{\; \text{sgn}}}
\newcommand{\UA}{\ensuremath{_{\uparrow}}}
\newcommand{\RA}{\ensuremath{_{\rightarrow}}}
\newcommand{\QED}{\left\{ \hfill{\textbf{QED}} \right\}}

%\newcommand{\diff}{%
%    \IfEqCase{frac{\diff}{%
%        {\ensuremath{frac{\text{d}} }}%
%        {\ensuremath{\; \text{d}} }% 
%    }[\PackageError{diff}{Problem with diff}{}]%
%}%


\date{\today}
\title{Diffusion simulation: Neurotransmitters in the synaptic cleft\\ \small{Project 4 -- FYS4150}}
\author{Marius Berge Eide \\
\texttt{mariusei@astro.uio.no}}


\begin{document}


\onecolumn
\maketitle{}

\begin{abstract}
    This project devises a way of solving a $N$-body gravitational problem using a fourth order Runge-Kutta differential equation solver. The procedures are implemented in a way that makes them easy to reuse, and are all dimensionless. The virial theorem is applied on Venus, and found to give that the total energy $2\langle K \rangle + \langle V \rangle = 0$ as there are no non-conservative forces acting on the system. Moon is found to orbit Earth with approximately 14 cycles per year. The perihelion precession of Mercury is reproduced, but is of five orders of magnitude smaller than the physical precession. The project only uses distances to Sun to calculate the initial velocities.
\end{abstract}

\section{Introduction}

\section{Methods}
\section{Results}
\section{Discussion and conclusion}


\bibliography{referanser}
\bibliographystyle{plain}


\end{document}

